\section*{Part A - Ransomware}

\noindent \begin{center}
\rule[0.5ex]{1\linewidth}{1pt}
\par\end{center}

\begin{comment}
    Ransomware has again come to the fore with the recent example of WannaCry in early May 2017. While
    what is did was not new, it managed to cause significant damage at a number of High profile organisations
    such as the UK NHS. Part of its 'success' is that it incorporated exploitation of the EternalBlue vulnerability
    within the SMB stack on Microsoft Windows platforms to spread, rather than they more traditional
    approach taken by ransomware operators of emailing out trojaned documents which contained the appropriate
    'dropper'. While it was initially expected that the many existing legacy installations of Windows
    XP and Windows Server 2003, both of which have been End of Life for some time were particularly hard
    hit. These systems were however not found to be the bulk of infections.
        a) How could organisations better prepare for future ransomware attacks. Bearing in mind that
        the UK NHS had been hit with other ransomware attacks in early 2017 [1,2,3]. Provide
        suitably motivated advice for a large organisation. (20)
        b) How would you adjust your advice for a smaller (<30 headcount) company? How do their
        risks differ from a large enterprise ? (10)
        c) In light of recent events, calls have been made for organisations to stockpile bitcoin as part of
        a DRP strategy [4,5,6]. Motivate for or against this opinion, providing suitable support for
        your argument. (15)
        d) One of the interesting observations with WannaCry is that it appears that relatively little
        money has changed hand in terms of ransom payments, with some organisations pitting this
        at around between 50K and 70K USD two weeks after launch [7]. This appears to have
        tailed off to around 150 000 USD [8,9]. Discuss this risk/reward tradeoff for an organisation
        running the WannaCry campaign. How does this compare to other data available on payments
        to ransomware operators. (15)
    
    The expected length 4-5 pages per question above, with Part A being 18-25 pages in total.
    
    [1] http://www.telegraph.co.uk/news/2017/01/13/largest-nhs-trust-hit-cyber-attack/
    [2] Cyber criminals target NHS to steal medical data for ransom
    https://www.ft.com/content/b9abf11e-e945-11e6-967b-c88452263daf
    [3] Ransomware brutes smacked 1 in 3 NHS trusts last year
    https://www.theregister.co.uk/2017/01/17/nhs_ransomware/
    [4] Companies Are Stockpiling Bitcoin to Pay O Cybercriminals
    https://www.technologyreview.com/s/601643/
    companies-are-stockpiling-bitcoin-to-pay-off-cybercriminals/
    [5] City banks plan to hoard bitcoins to help them pay cyber ransoms
    https://www.theguardian.com/technology/2016/oct/22/
    city-banks-plan-to-hoard-bitcoins-to-help-them-pay-cyber-ransoms
    [6] With the looming threat of Ransomware, should companies stockpile Bitcoins?
    https://www.bleepingcomputer.com/editorial/security/
    with-the-looming-threat-of-ransomware-should-companies-stockpile-bitcoins/
    [7] Hackers who infected 200,000 machines have only made \$50,000 worth of bitcoin http://www.cnbc.
    com/2017/05/15/wannacry-ransomware-hackers-have-only-made-50000-worth-of-bitcoin.html
    [8] WannaCry payment https://www.elliptic.co/wannacry/
    [9]http://wannacry.today/

    
\end{comment}

\cite{ISO27001:2013}
\cite{ISO27002:2013}